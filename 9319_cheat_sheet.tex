\documentclass[a4paper]{article}
\usepackage{amssymb,amsmath,amsthm,amsfonts}
\usepackage{multicol,multirow}
\usepackage{calc}
\usepackage{ifthen}
\usepackage[landscape]{geometry}
\usepackage[colorlinks=true,citecolor=blue,linkcolor=blue]{hyperref}


		\geometry{top=1cm,left=1cm,right=1cm,bottom=1cm} 
		\geometry{top=1cm,left=1cm,right=1cm,bottom=1cm} 
	
\pagestyle{empty}
\makeatletter
\renewcommand{\section}{\@startsection{section}{1}{0mm}%
                                {-1ex plus -.5ex minus -.2ex}%
                                {0.5ex plus .2ex}%x
                                {\normalfont\large\bfseries}}
\renewcommand{\subsection}{\@startsection{subsection}{2}{0mm}%
                                {-1explus -.5ex minus -.2ex}%
                                {0.5ex plus .2ex}%
                                {\normalfont\normalsize\bfseries}}
\renewcommand{\subsubsection}{\@startsection{subsubsection}{3}{0mm}%
                                {-1ex plus -.5ex minus -.2ex}%
                                {1ex plus .2ex}%
                                {\normalfont\small\bfseries}}
\makeatother
\setcounter{secnumdepth}{0}
\setlength{\parindent}{0pt}
\setlength{\parskip}{0pt plus 0.5ex}
% -----------------------------------------------------------------------

\begin{document}

\raggedright
\footnotesize

\begin{multicols}{3}
\setlength{\premulticols}{1pt}
\setlength{\postmulticols}{1pt}
\setlength{\multicolsep}{1pt}
\setlength{\columnsep}{2pt}

\subsubsection{Entropy}
$ \Sigma_{i = 1}^{n}{-p(s_{i})log_{2}{p(s_{i})}}$ 
\subsubsection{Arithmetic coding}
ENCODING \\
get an input symbol \\
code\_range = high - low \\
high = low + range*high\_range(symbol) \\
low = low + range*low\_range(symbol) \\
output minimum precision\\
DECODING \\
find symbol whose range straddles the encoded number\\
output the symbol\\
range = symbol high value - symbol low value\\
subtract symbol low value from encoded number\\
divide encoded number by range
\subsubsection{Dictionary coding LZW}
jejunojejuno\\
\begin{tabular}{c|c|c|c|c}
    w & k & output & index & symbol\\
    nil & j & & &\\
    j & e & j & 256 & je\\
    e & j & e & 257 & ej\\
    j & u & j & 258 & ju\\
    u & n & u & 259 & un\\
    n & o & n & 260 & no\\
    o & j & o & 261 & oj\\
    j & e &&&\\
    je & j & 256 & 262 & jej\\
    j & u & &&\\
    ju & n & 258 & 263 & jun\\
    n& o & 260 &&\\
\end{tabular}
\subsubsection{Adaptive Huffman}
Vitter's adaptive huffman:\\
For each weight w, all leaves precede internal nodes.
\subsubsection{Run Length BWT}
L: Original BWT; B: BWT in bit stream; B':Front Row in bit stream; S: Compact BWT(reduce repeat char);\\
\begin{tabular}{c|l}

Origin & BANANA\$\\
BWT & ANNB\$AA \\
B & 1 1 0 1 1 1 0\\
B' & 1 1 1 0 1 1 0 \\
F & \$AAABNN\\
S & \$ABN
\end{tabular}
\subsubsection{Xpath}
//author/*\\
// any level, * any element\\
/bib/book/@price, @price means that price is has to be an
attribute\\
/bib/book/author[firstname]\\
must have [] attribute as child\\
\subsubsection{ISX}
[open, close, forward max, forward min, back max, back min, no. text(leaf)]\\
find next close, go up then go down.
\end{multicols}

\end{document}
